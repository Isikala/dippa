\documentclass[12pt,a4paper,oneside,pdftex]{report}
\usepackage[utf8]{inputenc}
\usepackage[OT1]{fontenc}
\usepackage[finnish,swedish,english]{babel}
\usepackage[square,sort&compress,numbers]{natbib}
\usepackage{eurosym}
\usepackage{mathtools}
\usepackage{verbatim}
\usepackage{longtable}
\usepackage{subfigure}
\usepackage[medium]{titlesec}
\usepackage{tikz}
\usetikzlibrary{positioning}
\usetikzlibrary{calc}
\usetikzlibrary{arrows}
\usetikzlibrary{decorations.pathmorphing,decorations.markings}
\usetikzlibrary{shapes}
\usetikzlibrary{patterns}
\usepackage[mydraft,twosupervisors]{aalto-thesis}
%\usepackage[mydraft,doublenumbering]{aalto-thesis}
%\usepackage{aalto-thesis}
\RequirePackage[pdftex]{hyperref}
\hypersetup{colorlinks=false,raiselinks=false,breaklinks=true}
\hypersetup{pdfborder={0 0 0}}
\hypersetup{bookmarksnumbered=true}
% The following line suggests the PDF reader that it should show the
% first level of bookmarks opened in the hierarchical bookmark view.
\hypersetup{bookmarksopen=true,bookmarksopenlevel=1}
% Hyperref can also set up the PDF metadata fields. These are
% set a bit later on, after the thesis setup.


% Thesis setup
% ==================================================================
% Change these to fit your own thesis.
% \COMMAND always refers to the English version;
% \FCOMMAND refers to the Finnish version; and
% \SCOMMAND refers to the Swedish version.
% You may comment/remove those language variants that you do not use
% (but then you must not include the abstracts for that language)
% ------------------------------------------------------------------
% If you do not find the command for a text that is shown in the cover page or
% in the abstract texts, check the aalto-thesis.sty file and locate the text
% from there.
% All the texts are configured in language-specific blocks (lots of commands
% that look like this: \renewcommand{\ATCITY}{Espoo}.
% You can just fix the texts there. Just remember to check all the language
% variants you use (they are all there in the same place).
% ------------------------------------------------------------------
\newcommand{\TITLE}{Software Processes for Dummies:}
\newcommand{\FTITLE}{Ohjelmistoprosessit mänteille:}
%\newcommand{\STITLE}{Den stora stygga vargen:}
\newcommand{\SUBTITLE}{Re-inventing the Wheel}
\newcommand{\FSUBTITLE}{Uusi organisaatio, uudet pyörät}
%\newcommand{\SSUBTITLE}{Lilla Vargens universum}
\newcommand{\DATE}{June 18, 2011}
\newcommand{\FDATE}{18. kesäkuuta 2011}
%\newcommand{\SDATE}{Den 18 Juni 2011}

% Supervisors and instructors
% ------------------------------------------------------------------
% If you have two supervisors, write both names here, separate them with a
% double-backslash (see below for an example)
% Also remember to add the package option ``twosupervisors'' or
% ``twoinstructors'' to the aalto-thesis package so that the titles are in
% plural.
% Example of one supervisor:
%\newcommand{\SUPERVISOR}{Professor Antti Ylä-Jääski}
%\newcommand{\FSUPERVISOR}{Professori Antti Ylä-Jääski}
%\newcommand{\SSUPERVISOR}{Professor Antti Ylä-Jääski}
% Example of twosupervisors:
\newcommand{\SUPERVISOR}{Professor Antti Ylä-Jääski\\
  Professor Pekka Perustieteilijä}
\newcommand{\FSUPERVISOR}{Professori Antti Ylä-Jääski\\
  Professori Pekka Perustieteilijä}
\newcommand{\SSUPERVISOR}{Professor Antti Ylä-Jääski\\
  Professor Pekka Perustieteilijä}

% If you have only one instructor, just write one name here
\newcommand{\INSTRUCTOR}{Olli Ohjaaja M.Sc. (Tech.)}
\newcommand{\FINSTRUCTOR}{Diplomi-insinööri Olli Ohjaaja}
%\newcommand{\SINSTRUCTOR}{Diplomingenjör Olli Ohjaaja}
% If you have two instructors, separate them with \\ to create linefeeds
% \newcommand{\INSTRUCTOR}{Olli Ohjaaja M.Sc. (Tech.)\\
% Elli Opas M.Sc. (Tech)}
%\newcommand{\FINSTRUCTOR}{Diplomi-insinööri Olli Ohjaaja\\
% Diplomi-insinööri Elli Opas}
%\newcommand{\SINSTRUCTOR}{Diplomingenjör Olli Ohjaaja\\
% Diplomingenjör Elli Opas}

% If you have two supervisors, it is common to write the schools
% of the supervisors in the cover page. If the following command is defined,
% then the supervisor names shown here are printed in the cover page. Otherwise,
% the supervisor names defined above are used.
\newcommand{\COVERSUPERVISOR}{Professor Antti Ylä-Jääski, Aalto University\\
  Professor Pekka Perustieteilijä, University of Helsinki}

% The same option is for the instructors, if you have multiple instructors.
% \newcommand{\COVERINSTRUCTOR}{Olli Ohjaaja M.Sc. (Tech.), Aalto University\\
% Elli Opas M.Sc. (Tech), Aalto SCI}


% Other stuff
% ------------------------------------------------------------------
\newcommand{\PROFESSORSHIP}{Data Communication Software}
\newcommand{\FPROFESSORSHIP}{Tietoliikenneohjelmistot}
%\newcommand{\SPROFESSORSHIP}{Datakommunikationsprogram}
% Professorship code is the same in all languages
\newcommand{\PROFCODE}{T-110}
\newcommand{\KEYWORDS}{ocean, sea, marine, ocean mammal, marine mammal, whales,
cetaceans, dolphins, porpoises}
\newcommand{\FKEYWORDS}{AEL, aineistot, aitta, akustiikka, Alankomaat,
aluerakentaminen, Anttolanhovi, Arcada, ArchiCad, arkki}
%\newcommand{\SKEYWORDS}{omsättning, kassaflöde, värdepappersmarknadslagen,
%yrkesutövare, intresseföretag, verifieringskedja}
\newcommand{\LANGUAGE}{English}
\newcommand{\FLANGUAGE}{Englanti}
%\newcommand{\SLANGUAGE}{Engelska}

% Author is the same for all languages
\newcommand{\AUTHOR}{Mikael Siirtola}


% Currently the English versions are used for the PDF file metadata
% Set the PDF title
\hypersetup{pdftitle={\TITLE\ \SUBTITLE}}
% Set the PDF author
\hypersetup{pdfauthor={\AUTHOR}}
% Set the PDF keywords
\hypersetup{pdfkeywords={\KEYWORDS}}
% Set the PDF subject
\hypersetup{pdfsubject={Master's Thesis}}


% Layout settings
% ------------------------------------------------------------------

% When you write in English, you should use the standard LaTeX
% paragraph formatting: paragraphs are indented, and there is no
% space between paragraphs.
% When writing in Finnish, we often use no indentation in the
% beginning of the paragraph, and there is some space between the
% paragraphs.

% If you write your thesis Finnish, uncomment these lines; if
% you write in English, leave these lines commented!
% \setlength{\parindent}{0pt}
% \setlength{\parskip}{1ex}

% Use this to control how much space there is between each line of text.
% 1 is normal (no extra space), 1.3 is about one-half more space, and
% 1.6 is about double line spacing.
% \linespread{1} % This is the default
% \linespread{1.3}

% Bibliography style
% acm style gives you a basic reference style. It works only with numbered
% references.
\bibliographystyle{acm}
% Plainnat is a plain style that works with both numbered and name citations.
% \bibliographystyle{plainnat}


% Extra hyphenation settings
% ------------------------------------------------------------------
% You can list here all the files that are not hyphenated correctly.
% You can provide many \hyphenation commands and/or separate each word
% with a space inside a single command. Put hyphens in the places where
% a word can be hyphenated.
% Note that (by default) LaTeX will not hyphenate words that already
% have a hyphen in them (for example, if you write ``structure-modification
% operation'', the word structure-modification will never be hyphenated).
% You need a special package to hyphenate those words.
\hyphenation{di-gi-taa-li-sta yksi-suun-tai-sta}



% The preamble ends here, and the document begins.
% Place all formatting commands and such before this line.
% ------------------------------------------------------------------
\begin{document}
% This command adds a PDF bookmark to the cover page. You may leave
% it out if you don't like it...
\pdfbookmark[0]{Cover page}{bookmark.0.cover}
% This command is defined in aalto-thesis.sty. It controls the page
% numbering based on whether the doublenumbering option is specified
\startcoverpage

% Cover page
% ------------------------------------------------------------------
% Options: finnish, english, and swedish
% These control in which language the cover-page information is shown
\coverpage{english}


% Abstracts
% ------------------------------------------------------------------
% Include an abstract in the language that the thesis is written in,
% and if your native language is Finnish or Swedish, one in that language.

% Abstract in English
% ------------------------------------------------------------------
\thesisabstract{english}{

%TODO
}
% Abstract in Finnish
% ------------------------------------------------------------------
\thesisabstract{finnish}{
%TODO
}

% Acknowledgements
% ------------------------------------------------------------------
% Select the language you use in your acknowledgements
\selectlanguage{english}

% Uncomment this line if you wish acknoledgements to appear in the
% table of contents
%\addcontentsline{toc}{chapter}{Acknowledgements}

% The star means that the chapter isn't numbered and does not
% show up in the TOC
\chapter*{Acknowledgements}

%TODO

\vskip 10mm

\noindent Espoo, \DATE
\vskip 5mm
\noindent\AUTHOR

% Acronyms
% ------------------------------------------------------------------
% Use \cleardoublepage so that IF two-sided printing is used
% (which is not often for masters theses), then the pages will still
% start correctly on the right-hand side.
\cleardoublepage
% Example acronyms are placed in a separate file, acronyms.tex
% \input{acronyms}

\addcontentsline{toc}{chapter}{Abbreviations and Acronyms}
\chapter*{Abbreviations and Acronyms}

% The longtable environment should break the table properly to multiple pages,
% if needed

\noindent
\begin{longtable}{@{}p{0.25\textwidth}p{0.7\textwidth}@{}}
2k/4k/8k mode & COFDM operation modes \\
3GPP & 3rd Generation Partnership Project \\
ESP & Encapsulating Security Payload; An IPsec security protocol \\
FLUTE & The File Delivery over Unidirectional Transport protocol \\
e.g.& for example (do not list here this kind of common acronymbs or abbreviations, but only those that are essential for understanding the content of your thesis. \\
note & Note also, that this list is not compulsory, and should be omitted if you have only few abbreviations

\end{longtable}


% Table of contents
% ------------------------------------------------------------------
\cleardoublepage
% This command adds a PDF bookmark that links to the contents.
% You can use \addcontentsline{} as well, but that also adds contents
% entry to the table of contents, which is kind of redundant.
% The text ``Contents'' is shown in the PDF bookmark.
\pdfbookmark[0]{Contents}{bookmark.0.contents}
\tableofcontents

% List of tables
% ------------------------------------------------------------------
% You only need a list of tables for your thesis if you have very
% many tables. If you do, uncomment the following two lines.
% \cleardoublepage
% \listoftables

% Table of figures
% ------------------------------------------------------------------
% You only need a list of figures for your thesis if you have very
% many figures. If you do, uncomment the following two lines.
% \cleardoublepage
% \listoffigures

% The following label is used for counting the prelude pages
\label{pages-prelude}
\cleardoublepage

%%%%%%%%%%%%%%%%% The main content starts here %%%%%%%%%%%%%%%%%%%%%
% ------------------------------------------------------------------
% This command is defined in aalto-thesis.sty. It controls the page
% numbering based on whether the doublenumbering option is specified
\startfirstchapter

% Add headings to pages (the chapter title is shown)
\pagestyle{headings}

% The contents of the thesis are separated to their own files.
% Edit the content in these files, rename them as necessary.
% ------------------------------------------------------------------

% \input{1introduction.tex}

\chapter{Introduction}
\label{chapter:intro}

%testiporausten merkitys kaivosteollisuudelle

%pitkä analysointiajat -> joskus voidaan tehän liikaa koeporauksia -> kallista

%parannettu onsite analyysi -> lyhyemmät vasteajat -> parempi ohjaus koeporauksiin

%diplomityön tavoitteet
%tutkia suskismittauksia porasydämmistä suoraan laatikosta 
%ensin simulointi -> sitten protot


% \input{2background.tex}

\chapter{Background}
\label{chapter:background}

This chapter introduces different core logging methods and the coring process. 
Magnetic susceptibility measurements are discussed in dept.

\section{Core Logging}

Diamond drilling is commonly used method used for gaining core samples. It is expensive but highly
mobile and easy to prepare method. Diamond drill has an inner tube that collects the core and it can 
usually  hold a core lenght of 1,5 to 3 meteres. When the inner tube is filled with rock it is brought
back to surface with wires. Diamond drilling is slow on very hard rocks and collecting a whole core
sample on soft stones can prove to be a challenge. Rotary Drilling is cheaper and faster than diamond
drilling but it does not give a core to be examined later and can only be used for making bore holes
for well logging.\cite{Peters1987}

After core has been extracted from the rock it is quickly given first analysis by geologist with 
hand held lens or microscopes to guide the drilling process before more time consuming studies. 
Next the core is divided into practical lenghts and split into two
sections either by mechanical splitter, chisel or diamond saw. Half is taken for geological further
logging and the other one is stored for metallurgy and assay. Some of the core may be left whole
for geomechanical studies and split later. In laboratory the core is examined for major 
stuctures and lithological zone. Then it will be analysed and sampled. The sampling resoultion 
is nontrivial because it is time consuming and high resolution results in longer feedback time.\cite{Peters1987}

There are two kinds of core analysis methods, destructive and nondestructive. In destructive methods
are irreversible as it effects the structure of the core sampled and limits the methods that can be used
afterwards. Nondestructive methods do not affect the struckture of the core so the do not limit the 
analyses that can be conducted after them. Nondesctuctive methods usually provide good oversight of the
sample before more accurate desctrucktive methods.

'Wet' analysis is the most traditional of analysis methods. It is destructive as core is sampled and 
ground into fine powder. %TODO erilaiset wet analysis methodit tähän kirjasta!

Old X-ray methods used to be destructive \cite{Croudace2006} but newer methods for 
example one used in ITRAX \cite{Croudace2006} are not. ITRAX is half core analysis scanner where core half is 
driven by a motor through an X-ray beam and then optical and X-radiographic values are recorded. 
X-ray beam is intensively focused through a flat capillary waveguide to irradiate the core half. 
This provides resolution up to 200 $\mu$m.%tätä vielä tarkemmin ei kuitenkaan hirveen paljoa

%-infrared analysis
%-muut analyysit

\section{Magnetic Susceptibility Measurements}

Magnetic susceptibility is by definition magnetization divided by the field it is applied with.
Mathematically it can be expressed as  $\kappa$ that can be aguired by dividing induced magnetization 
per volume unit of measured sample (M) with applied magnetic field intensity (H)\cite{Lascu2009}:
\begin{equation}
$\kappa$ = $\frac{M}{H}$
\end{equation}
$\kappa$ has no dimensions since M and H have the same unit A/m. However if we know density of the 
material($\rho$) we may calculate mass specific magnetic susceptibility with following\cite{Lascu2009}:
\begin{equation}
$\chi$ = $\frac{\kappa}{\rho}$
\end{equation}
All substances may be categorised by their magnetic properties. Ferromagnetic minerals remember 
magnetic field that was applied on them by being able to contain a magnetic remnance. Good examples of 
ferromagnets are oxides such as magnetite (Fe$_3$O$_4$), hematite (Fe$_2$O$_3$) and sulfides like pyrrhotite (Fe$_7$S$_8$) and greigite (Fe$_3$S$_4$). Ferromagnets have strongly 
positive magnetic susceptibility. Paramagnetic materials allow only weak interaction between metal atoms. Their
magnetic susceptibility is weakly positive. Paramagnetic minerals include Fe-Ti oxides (illmenite), Fe-Mn carbonates 
(siderite, rhodochrosite) and Fe sulfides (e.g. pyrite). Dimagnetic substances posses negative magnezitation when 
they are applied a weak magnetic field. They are minerals that do not contain iron, ogranic materials and artificial 
materials such as polyethylene and polycarbonates. According to Lascu\cite{Lascu2009}  dimagnetic materials are 
especially important because paleolimnology workers sometimes mistake errors caused by dimagnetic materials to be 
caused by errors in the instruments. \cite{Lascu2009}

Magnetic susceptibility can be measured easily from both whole core and half core samples. The two most popular 
measurement apparatus types for continous measurements are loop sensors and point sensors. Loop sensors are loop
like systems and they measure the magnetic susceptibility by the change in magnetic flux when full core is move
through them. The best performance is acquired when loop sensor has radius close to the core radius. According to the 
research\cite{Nowaczyk2001}: it is also important to consider the results at the begging of core section and at the
end of it. The core affects the magnetic field of the coil before it enters it and the air behind the core also has 
an effect of the readings at the end of the core section. In the research\cite{Nowaczyk2001}: a way to fix this 
error was proposed. Following core parts may be attached together so there is no gap between them. This might be
problematic because sometimes core parts have caps in their both ends thus preventing this. Other method proposed
in this research\cite{Nowaczyk2001}: was to start logging well before the core and continue the logging after the
core has passed the sensor. Good results may be acquired by superimposing the core logs after the previous core with
core log of the following in respect to a common dept scale.
core part . This would leave only small part of the measurement at the beginning and at
end of the core sample air-biased \cite{Nowaczyk2001}:. In this method it is important that the common dept scale
of the cores is correct. Otherwise articial peaks may be noted in measurements.


%miten suskista käytännössä mitataan
%kuvat laitteista
%etc. beenis


%gabbaleen vaihto 8=========D
%kuuluuko aihe alueeseen?
%ilmesesti ei

%Petrofysics is combination of geology and study of how rocks interact with liquids. 
%It helps us understand how reservoirs filled with hydrocarbons (gases, oil) are formed.
%Reservoirs are complex three-dimensional strucktures filled with pores.

%-porosity

%-permeability

%-resistivity

%-water saturation

%-wettability

% \input{3environment.tex}

\chapter{Environment}
\label{chapter:environment}


% \input{4methods.tex}

\chapter{Methods}
\label{chapter:methods}

% \input{5implementation.tex}

\chapter{Implementation}
\label{chapter:implementation}

% \input{6evaluation.tex}

\chapter{Evaluation}
\label{chapter:evaluation}

% \input{7discussion.tex}

\chapter{Discussion}
\label{chapter:discussion}

% \input{8conclusions.tex}

\chapter{Conclusions}
\label{chapter:conclusions}

% Load the bibliographic references
% ------------------------------------------------------------------
% You can use several .bib files:
% \bibliography{thesis_sources,ietf_sources}
\bibliography{ref}


% Appendices go here
% ------------------------------------------------------------------
% If you do not have appendices, comment out the following lines
\appendix
% \input{appendices.tex}

\chapter{First appendix}
\label{chapter:first-appendix}

This is the first appendix. You could put some test images or verbose data in an
appendix, if there is too much data to fit in the actual text nicely.

For now, the Aalto logo variants are shown in Figure~\ref{fig:aaltologo}.

\begin{figure}
\begin{center}
\subfigure[In English]{\includegraphics[width=.8\textwidth]{aalto-logo-en}}
\subfigure[Suomeksi]{\includegraphics[width=.8\textwidth]{aalto-logo-fi}}
\subfigure[Pä svenska]{\includegraphics[width=.8\textwidth]{aalto-logo-se}}
\caption{Aalto logo variants}
\label{fig:aaltologo}
\end{center}
\end{figure}


% End of document!
% ------------------------------------------------------------------
% The LastPage package automatically places a label on the last page.
% That works better than placing a label here manually, because the
% label might not go to the actual last page, if LaTeX needs to place
% floats (that is, figures, tables, and such) to the end of the
% document.
\end{document}